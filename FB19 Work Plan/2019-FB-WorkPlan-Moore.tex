\documentclass[14pt,english,letterpaper]{scrartcl}
\usepackage[top=1in, bottom=1in, left=1.25in, right=1.25in]{geometry}
\usepackage{graphicx}
\usepackage[unicode=true,pdfusetitle,
bookmarks=true,bookmarksnumbered=false,bookmarksopen=false,
breaklinks=true,pdfborder={0 0 1},backref=false,colorlinks=true]
{hyperref}
\hypersetup{urlcolor=blue,linkcolor=blue}
\usepackage{pdfpages}
\usepackage{indentfirst}

\title{\small COOPERATIVE AGREEMENT WORK PLAN\\
Plant Protection and Quarantine, Science and Technology\\	
and\\
University of Guam College of Agriculture and Life Sciences\\
{\bigskip}
\Huge Coconut Rhinoceros Beetle Biocontrol}

\author{Aubrey Moore}

\begin{document}

\maketitle
\newpage
\tableofcontents
\pagebreak

%\textbf{Period of Performance:} (time from the beginning to the end of
%the agreement\ldots{}this should be specific dates such as December 01,
%2016 to November 30, 2017.)

\textbf{Performance period:} August 1, 2019 to July 31, 2020

\section{Introduction}

%This should include a paragraph to
%identify the cooperating parties and the overall purpose of the
%initiative as illustrated in the paragraph below.{]}
%
%This Work Plan reflects a cooperative relationship between (Institution)
%and Animal and Plant Health Inspection Service (APHIS), Plant Protection
%and Quarantine (PPQ). It outlines the mission-related goals, objectives,
%and anticipated accomplishments as well as the approach for conducting
%(insert a statement that describes the program or research being
%conducted) and the related roles and responsibilities of the parties as
%negotiated.

A newly discovered biotype of \textit{Oryctes rhinoceros}, coconut rhinoceros beetle (CRB-G), is rapidly
killing coconuts and other palms on Guam. Uncontrolled outbreaks of CRB-G are also occurring
in Papua New Guinea, Solomon Islands, and Palau. Eradication is being attempted on Oahu and
Rota. Following a failed eradication attempt on Guam, CRB-G proved hard to control because
this biotype is resistant to \textit{Oryctes rhinoceros} nudivirus (OrNV), which was previously used as a
very effective biological control agent for control of CRB outbreaks on Pacific Islands and
elsewhere.

The primary objective of this proposed project is to stop the uncontrolled outbreak on Guam.
Pacific-based entomologists working on the CRB-G problem agree that the most feasible
solution is to find and release a new isolate of OrNV which is highly pathogenic to CRB-G. All
previous OrNV releases on Pacific Islands prior to the invasion of Guam by CRB-G resulted in
immediate and sustained suppression of CRB damage to low levels and prevention of tree
mortality. We hope to find an OrNV isolate which will produce similar results. Our plan is to
search for effective OrNV isolates which are controlling Asian populations of CRB-G. To date,
CRB-G has been detected in the Philippines, Indonesia, Thailand, Taiwan and Japan. Foreign
exploration for an effective OrNV isolate began in 2017 with an expedition to Negros Island in
the Philippines. OrNV infecting a single CRB-G adult was isolated, but we have been unable
to infect CRB-G adults collected on Guam with this isolate in laboratory bioassays. When an
effective OrNV isolate is found, it will be propagated \textit{in vivo} and released into the Guam CRB-G
population under the terms of an existing USDA-APHIS import and release permit.

A secondary objective is to establish an island-wide monitoring system to track temporal and
spatial changes in the extent of CRB damage to Guam’s coconut palms. Damage symptoms such
as v-shaped cuts to fronds, bore holes, and dead standing coconut palm stems are readily
observed during roadside surveys. Survey data will be collected using a digital video
camera mounted on a truck. Initially, video images of coconut palm damage by CRB-G will be
detected, classified and tagged by a technician. When a large number of images have been
tagged, these will be used to train an automated CRB damage detection and monitoring
system using computer vision and deep learning. This automated system will be useful for monitoring results of biocontrol and other control activities. It may also be used as an early detection device for CRB.

\section{Background} 

%(What relevant need or problem within the
%cooperator's mission area requires a solution in carrying out a public
%purpose of support or stimulation? - This section includes a narrative
%on how financial assistance will facilitate the cooperator in carrying
%out a public purpose of support or stimulation. It is recommended that
%the first sentence of this section begin as stated below.)
%
%The primary purpose of this agreement is to support (state the public
%purpose).

The primary purpose of this agreement is to stop an uncontrolled outbreak of CRB-G on Guam using biological control.

Coconut rhinoceros beetle (CRB), \textit{Oryctes rhinoceros}, is a major pest of palms. Adults bore into
crowns to feed on sap. A palm may be killed if CRB feeding activity damages the meristem, but
this rarely happens at low CRB population densities. CRB grubs do no damage. They feed on
decaying vegetation with standing dead coconuts and fallen coconut logs being favored food source. In
addition, they can feed in many types of organic matter including dead trees, green waste, saw
dust, manure, compost, and even in bags of commercially packaged soil.

CRB was first detected on Guam in 2007. An eradication attempt using mass
trapping and sanitation failed when the beetle spread to all parts of the island within a few years. Following this failure, \textit{Oryctes
rhinoceros} nudivirus (OrNV) and green muscardine fungus (GMF), \textit{Metarhizium majus}, where
introduced as biological control agents. GMF was successfully established as a classical biocontrol agent and a 2015 survey
indicated that between 10\% and 38\% of Guam's CRB were infected by this fungus. However, the
preferred biocontrol agent for CRB, namely OrNV, failed to have any effect. This lead us to
discover that the Guam CRB population is genetically distinct from other Pacific island populations of this major palm pest and it is being
referred to as the CRB-G biotype. CRB-G is resistant to all available isolates of OrNV, previously
the most effective biocontrol agent for CRB, and it appears to have other characteristics, which
make it more invasive and harder to control than other CRB biotypes. While there where no
range expansions of CRB for a quarter of a century (1980 to 2005), CRB is now on the move with
invasion of Guam in 2007, the Port Moresby area of Papua New Guinea in 2009, Oahu,
Hawaii in 2013, and the Honiara area of Guadalcanal, Solomon Islands in 2015, and Rota in
2017. It is significant that all of these new invasions involve CRB-G. Thus, CRB-G is a regional
problem which poses significant risks to Pacific island economies and ecosystems.

The current, full-on CRB-G outbreak on Guam was triggered by Typhoon Dolphin which visited
the island in May 2015. It was not a very strong typhoon by Guam standards, but it was the first
one in more than a decade and it created a lot more damage than expected. Abundant piles of
decaying vegetation became CRB breeding sites. Some of these new breeding sites were in
villages were they could be managed. But most were inaccessible: in jungles and/or on military
land. Within a few months, massive numbers of adults were emerging from breeding sites and
severely attacking palms which started to die. Prior to Dolphin, we saw some heavily CRB-
damaged palms,but very few dead ones. A self-sustaining positive feedback cycle began
whereby large numbers of adult beetles attacked and killed large numbers of palms which
became breeding sites which generated even higher numbers of adults. Severe damage to
Guam’s palms prompted the Governor of Guam to declared a state of emergency in July 2017. If
the Guam CRB-G infestation cannot be controlled, it is expected that most palms on the island
will be killed and CRB-G will spread to other islands and beyond. If CRB-G invades smaller
islands and atolls where coconut is the tree of life, a human tragedy will ensue. On larger
islands invaded by CRB-G, coconut and oil palm industries, tourism, and native ecosystems are being severely
impacted

Concerned Pacific-based entomologists are attempting to raise support for coordinated regional
response to CRB-G. APHIS supported this effort by hosting a meeting at the International
Congress of Entomology, Florida, 2016. Another meeting aimed at organizing a formal
collaboration in response to CRB-G was part of the 2018 International Congress of
Invertebrate Pathology and Microbial Control and the 51st Annual Meeting of the Society for
Invertebrate Pathology which was held on the Gold Coast, Queensland, Australia in August 2018.
Future meetings specifically addressing international collaboration on CRB-G biocontrol include
one at the International Association of Plant Protection Sciences meeting, Hyderabad, India 2019 and
one tentatively being planned in association with the Pacific Plant Protection Organization on Guam during 2020.

\pagebreak

Financial assistance for this project will provide:
\begin{itemize}
	
	\item continued support for operation of an insect pathology laboratory at the University of
	Guam to evaluate candidate biocontrol agents discovered during foreign exploration
	
	\item continued support for a graduate research assistant at the University of Guam
	
	\item support for establishment and operation of island-wide CRB damage surveys on Guam
	
	\item continued support of an international collaborative project with the goal of discovering	an isolate of OrNV or other microbial biocontrol agent which can be used for long-term suppression of CRB-G populations

\end{itemize}

\paragraph{Project Staff}

Funds from this grant are requested to support a graduate assistant and 3 part-time student workers.
Collaboration with Dr. Sean Marshall, an insect pathologist at AgResearch New Zealand who is recognized as a world expert on biocontrol of CRB with OrNV, will be supported via a contract with his institution. Participation of other members of the project team is supported by other grants.

\begin{itemize}
\item PI: Aubrey Moore, PhD; Insect Ecologist
\item James Grasela, PhD; Insect Pathologist; funded for 2 years by a grant from Dept. of
Interior, Office of Island Affairs
\item Ian Iriarte, BS; Graduate Assistant
\item Roland Quitugua, MS; Plant Pathologist; Collaborator
\item Sean Marshall, PhD; Insect Pathologist; Collaborator; Participation funded by a contract
between the University of Guam and AgResearch New Zealand
\item 3 part-time student workers will assist with laboratory and field activities
\end{itemize}

\pagebreak

\section{Goals and Objectives} \label{goals}

%(You may use one or the other if
%desired) (List or explain what results or benefits will be derived from
%the cooperative effort? These will be the major building blocks upon
%which the milestones in the next section are based.)

\subsection{Objective 1: CRB Biocontrol}

The primary objective is to find an OrNV isolate which is highly effective biological control agent for long-term suppression of CRB-G populations. As soon as laboratory studies indicate discovery of a OrNV isolate which is a potential biological control agent for CRB, we will multiply the virus \textit{in vivo} and initiate field releases under the conditions of an existing USDA-APHIS permit.

\subsubsection{Regional Collaboration}

Work will continue to work with colleagues at AgResearch New Zealand, the Secretariat of the Pacific Community (SPC), Tokyo University, the University of Hawaii and others to put together a regional collaboration with the objective of finding an effective biocontrol agent for CRB-G. 

\paragraph{Methods}
\begin{itemize}
\item Moore and Grasela will participate in the CRB-G biocontrol meeting at the IAPPS meeting in Hyderabad, India, November 2019.
\item Moore will continue to maintain a web site to facilitate exchange of information on CRB-G biocontrol.
\end{itemize}

\subsubsection{Foreign Exploration for an Effective Biocontrol Agent for CRB-G}

Foreign exploration in search of a microbial biocontrol agent for CRB-G is already underway.
During January, 2017, Moore, Iriarte and Marshall collected an isolate if OrNV from a CRB-G population in Negros Island, Philippines. Laboratory bioassays indicate that this isolate is not a good candidate for biocontrol.

We are currently performing laboratory bioassays to evaluate two novel isolates obtained from AgResearch New Zealand.

In addition we are attempting to isolate OrNV from CRB adults collected in Taiwan. Dr. Shizu Watanabe, University of Hawaii,reported an 82\% OrNV infection rate in CRB-G collected from this island.

Our next target population is CRB-G found on the southern islands of Japan. We plan to collaborate with Dr. Madoka Nakai, Tokyo University of Agriculture and Technology, to obtain CRB-G/OrNV specimens from these islands.

\paragraph{Methods}

\begin{itemize}
	\item Subsamples of CRB collected during foreign exploration will be shipped to AgResearch New Zealand to determine CRB biotype and to isolate OrNV
	\item OrNV isolates will be tested in the insect pathology lab at UOG using standard bioassay protocols.
\end{itemize}

\subsubsection{Establish Lab Colonies of CRB-G and CRB-S}

We will establish sustainable laboratory colonies of CRB-G and virus susceptible beetles (CRB-S) as a source of healthy beetles for bioassays. 

Note: Establishment of a CRB-S colony is contingent on receiving a USDA-APHIS import permit to import live coconut rhinoceros beetles. I requested a permit on March 19, 2019 (Application number P526-190319-001) to replace a previous permit, P526P-11-01844, which I accidentally allowed to lapse into oblivion after only one shipment.

If we are allowed to import CRB-S, this will allow us to do comparative studies to:
\begin{itemize}
	\item Measure difference in susceptibility to OrNV isolates. (Resistance of CRB-G to OrNV has not yet actually been proven by comparative bioassays.)
	\item Test for behavioral differences. (It has been hypothesized that the aggregation pheromone, oryctalure, is less attractive to CRB-G than CRB-S.)
\end{itemize}

\paragraph{Methods}

\begin{itemize}
	\item Larvae and adults will be reared individually in Mason jars enclosed by metal caps. Larvae will be fed a store-bought steer manure/soil blend on which we have reared CRB larvae for many years. Adults will be bedded in peat moss and fed banana slices weekly. 
	\item Mason jars will be placed in enviromnmental cabinets set at 30 deg. C, 80\% RH and 12 h photoperiod. 
	\item Detailed records for each individual beetle will be stored in an existing online laboratory information management system (LIMS). These data will be made available to USDA-APHIS.
	
\end{itemize}


\subsubsection{Establish Sustainable CRB-G Biocontrol CRB-G by Autodissemination}

When bioassays indicate that an OrNV isolate is a potential biocontrol candidate, the virus will be
propagated in vivo and released into the Guam CRB-G population by autodissemination. Autodissemination involves infecting healthy CRB adults with OrNV. These infected beetles are then released at points dispersed throughout the island where they vector disease to conspecifics. 

\paragraph{Methods}

\begin{itemize}
\item On Guam, beetles for \textit{in vivo} propagation and autodissemination will be field-collected from
breeding sites and pheromone traps because this is far more efficient than rearing beetles in the lab at
the current time.

\item Concurrent with autodissemination releases, laboratory bioassays will be performed to quantify the toxic
(LD50, LT50, etc.) and nontoxic effects (fecundity, flight capability, etc.) of OrNV on CRB-G.

\item  There will also
be an attempt to increase virulence by cycling isolates through several generations of beetles. \end{itemize}

\subsection{Objective 2: Establish a Sustainable Coconut Palm Health Monitoring System}

The CRB-G outbreak on Guam is currently unmonitored on an island-wide basis. An island-wide
pheromone trapping system, using about 1500 traps, was operated by the University of Guam from 2008
to 2014. This monitoring system was transferred to the Guam Department of Agriculture which
abandoned the effort at the end of February, 2016. 

Currently, many coconut palms are being killed by
CRB-G. But, in the absence of a monitoring system, we do not have an estimate of tree mortality or
whether or not the damage is increasing or decreasing.
Clearly, establishment of a monitoring system is necessary if we want to evaluate success of the
proposed biocontrol project, or any other mitigation efforts. 

Rather than re-establish a trapping survey, we intend to establish a monitoring system
to track temporal and spatial changes in the extent of CRB damage to Guam’s coconut palms. Damage
symptoms such as v-shaped cuts to fronds, bore holes, and dead standing coconut palm stems are
readily observed during roadside surveys. Survey data will be collected using a digital video
camera mounted on a truck. Initially, video images of coconut palm damage by CRB-G will be detected,
classified and tagged by a technician. When a large number of images have been tagged, these will be
used to train a fully automated CRB damage detection and monitoring system. This automated system
may be useful as an early detection device for CRB. Roadside surveys on Guam will be performed
bimonthly.

\paragraph{Methods}

\begin{itemize}
	
	\item A protocol will be developed to perform roadside surveys of CRB damage. Damage will be recorded using videos recorded by a vehicle-mounted Olympus TG-5 camera. This camera records GPS coordinates.
	
	\item Videos will be tagged using the open source Computer Vision Annotation Tool (CVAT).
	
	\item An object detector which locates and classifies CRB damage in video recordings will be trained using annotated videos from the previous step. We intend to use the TensorFlow implementation of the Faster R-CNN Deep Learning model. Training a CRB damage detector using deep learning requires use of a computer with specialized software (TensorFlow) and specialized hardware (a graphics processing unit (GPU)). Instead of purchasing a physical machine we will rent a virtual machine designed specifically for this application from Amazon Web Services.
	  
	\item Results from the trained object detector will be evaluated using the human annotated videos.
	
	\item We will develop an automated processing system which takes roadside videos as input and generates CRB damage maps as output.
\end{itemize}

\section{Milestones} 

%(Milestones should list incremental steps of
%achievement that successfully completes a goal and/or objective listed
%in section III and \emph{should be associated with a timeline}.)

\subsection{Objective 1: CRB-G Biocontrol}

\begin{description}
	
	\item[Month 1,2,3:] Complete laboratory bioassays of all new OrNV isolates currently on hand.
	
	\item[Month 2:] Establish a breeding colony for CRB-G and CRB-S.
	
	\item[Month 4:] Collect CRB-G from Japan.
	
	\item[Month 5,6,7:] Perform bioassays on Japanese OrNV isolate.
	
	\item[Month TBD:] Upon discovery of an OrNV isolate which has potential for use as an effective biological control agent, establish in vivo production and intitiate releases.
	
	\item[Month 12:] Prepare final report.
\end{description}


\subsection{Objective 2: Establish a CRB Damage Monitoring System}

\begin{description}
	
	\item[Months 1,3,5,7,9,11:] Island-wide roadside surveys will be done bimonthly using images recorded using an Olympus TG5 camera equipped with a GPS receiver mounted on a vehicle.

	\item[Month 1:] Annotate videos using the Computer Vision Annotation Tool (CVAT).
	
	\item[Month 2:] Use annotations to train an object detector for CRB damage using the Faster R-CNN model implemented in the TensorFlow object detection API. This work requires setting up a virtual machine with GPU on AWS.
	
	\item[Month 3:] Evaluate results from the trained object detector. If precision is insufficient, collect more annotated videos and add these to the training set.
	
	\item[Month 4:] Develop a software system which will take raw video GPS tracks as input, outputting CRB damage maps and statistics.
	
	\item[Month 12:] Prepare final report.

\end{description}

\section{Methods} 

%(This section describes the plan of action or
%approach to the work. Depending on the content of the two preceding
%sections and the nature of the agreement you may or may not need to
%include this section. If the agreement is not research oriented you may
%want to label this section \textbf{Plan of Action}.)

Methods are include with each objective listed in section(\ref{goals}).

\section{Deliverables} 

%(This section should clearly state what is to
%be delivered at the end of the agreement. The accomplishments should be
%supportive of the primary purpose statement in section II. above. If a
%report, this report should be more comprehensive and state not only the
%accomplishments of the work but also the benefits of the accomplishments
%to the stakeholder.)

\begin{itemize}
	
	\item Foreign exploration leading to discovery of a highly pathogenic strain of OrNV or other microbial
	biocontrol agent for CRB-G could lead to implementation of self sustaining population
	suppression and tolerable damage levels on Guam and other islands invaded by CRB-G.
	
	\item Loss of 50\% or more of Guam’s palms may be prevented if an effective biocontrol agent is found
	and released quickly.
	
	\item Reduction in CRB population levels on Guam will reduce the risk of accidental of accidental transport of the highly
	invasive CRB-G biotype to other Pacific islands and elsewhere.
	
	\item Development of image analysis methods may lead to a small, inexpensive, automated CRB
	damage detector which could be mounted on a drone or a conventional vehicle. This
	device could be used for early detection or monitoring of CRB damage

\end{itemize}

\section{Resources} 

%(This section is for the purpose of explaining
%and justifying the funds listed in the budget section below. Negative
%reporting is not necessary. If you are not funding any area listed below
%in the budget, do not list or address that area in this discussion.)

\subsection{Salary and Benefits} 

%(What numbers and types of personnel
%will be needed and what role will they play in the execution of the
%agreement. How much time each position will contribute to the agreement
%(FTE, \%, \# Hours) Benefits are typically listed as a percentage of
%salary.) The following is an example of how this might read:
%
%
%``One Principle Investigator to assist in project planning, oversight
%and reporting (80\% time for 12 months). Salary (\$40,000) and benefits
%at 41.1\% (\$16,560). Subtotal: \$56,560.
%
%One Postdoctoral Research Associate at 1 FTE for 12 months requires full
%salary (\$49,440) and benefits at 35.58\% (\$17,592) for information
%collection, data analysis and document preparation. Subtotal: \$67,032.
%
%``One Research Assistant at 34\% time for 9 months requires partial
%salary (\$16,966) and benefits at 35.38\% (\$6,002) to support data
%collection, assessments, and analysis. Subtotal: \$22,968.
%
%Salary for 2 temporary Lab Technician for sample processing at \$2,400
%each (160 hrs. @ \$15/hr each and benefits at 9.65\% (\$232 each).
%Subtotal: \$5,264.'')

\begin{description}
	\item[Graduate Assistant] Ian Iriarte, BS; Graduate Assistant: \textbf{\$35,000} per annum plus 27\% benefits
	
	\item[Student Employees]  4 part-time student employees to assist with laboratory and field activities 4*700h*\$15/h=\textbf{\$42,000} plus 27\% benefits
	
	\item[Salary Reimbursement for PI] Aubrey Moore; 10\% FTE @ \$90,000 = \textbf{\$9,000}
	
	\item[Total Salaries] \textbf{\$86,000}
		
	\item[Total Benefits] 0.27x(Total salaries for graduate assistant and student employees) \textbf{\$20,790}
		
	\item[Total Salaries and Benefits]  \textbf{\$97,790}
			
\end{description}

\subsection{Supplies} 

%What supplies will be purchased to perform the
%work? Identify individual supplies with a cumulative value of \$5,000 or
%more as a separate item. All information technology supplies (e.g.,
%small items of equipment, connectivity through air cards or high speed
%internet access, radios for emergency operations) should be specifically
%identified.

\begin{description}
	
	\item[Cloud computing] Training a CRB damage detector using deep learning requires use of a computer with specialized software (TensorFlow) and specialized hardware (a graphics processing unit (GPU)). Instead of purchasing a physical machine we will rent a virtual machine designed specifically for this application from Amazon web services. We expect to run this machine for a total of 30 days. Estimated cost is \textbf{\$648} (30 d x 24 h x \$0.90 per h).
		
	\item[Chemicals and reagents] \$2,500
	
	\item[Microinjector] \$2,747
	
	\item[pH meter]	\$580
	
	\item[media for rearing beetles]	\$2,656
	
	\item[Vehicle fuel and maintenance]	\$5,000
	
	\item[Computers and computer supplies]	\$5,000
	
	\item[Miscellaneous lab supplies]	\$745
	
	\item[Total Supplies] \textbf{\$19,876}
	
\end{description}

\subsection{Travel} 

%(All travel provide for in the budget should be
%identified. This would include local travel to work sites and extended
%overnight travel both domestic and foreign. Identify travel to
%conferences and meetings separate from other travel. Travel to
%conferences and meetings should be for the purpose of presenting on the
%work of the agreement. The following is an example of how this might
%read:
%
%``Domestic: Costs for travel and lodging in Beltsville NC for PI and
%Research Assistant to collect/ screen samples (Flight (\$800), 2 nights
%lodging (\$200 per night), 3 days per diem (\$112 per day)). Subtotal:
%\$1,536.''
%
%1 Conference Trip, American Society of Horticultural Science in Hawaii
%for PI and Grad Student (Airfare \$500, Hotel 2 Nights, \$418, Taxi
%to/from Airport \$100). Subtotal: \$1,936.''
%
%Weekly travel to the field sites (San Diego, Orange, Los Angeles,
%Ventura and Santa Barbara) we will need a truck (rented from the
%University at \$354.61 per month for 8 months) at a cost of \$ 0.535 per
%mile. \$4,977.''
%
%International: PI and Research Assistant are planning to visit New
%Zealand once to develop innovative methods to evaluate uncertainty
%associated with pest forecasts through collaboration with Lincoln
%University (Lincoln, New Zealand). (Flight (\$3500), 5 nights lodging
%(\$2000 per night), 6 days per diem (\$672+ per day)). Subtotal:
%\$6,172.''
%
%\textbf{PLEASE NOTE -- Statements such as ``the PI will determine the
%appropriate time, duration, and personnel needed for travel'' will not
%be accepted.})

\begin{description}
	
	\item[International] PI and Post Doc plan to participate in a special meeting on biocontrol of CRB-G at the International Association of Plant Protection Science meeting in Hyderabad, India, November 10-14, 2019.
Estimated cost per person is \textbf{\$5118} (Registration: \$760; Airfare: \$2048; Per diem: 7d@\$330=\$2310).

	\item[International] PI and Post Doc plan a one week trip to Japan to work in collaboration with Dr. Madoka Nakai, Tokyo University of Agriculture and Technology, to collect isolates of OrNV from CRB-G populations in that country. January 15-22, 2020. Estimated cost per person is \textbf{\$4099}. (Airfare: \$550; Per diem: 7d@\$507=\$3549) 
	
	\item[Total Travel] 2x(\$5118+\$4099)=\textbf{\$17,334}
			
		
\end{description}

\subsection{Contracts or Sub-Agreements} 

%(Any plans for contracts or
%sub-agreements should be listed and an explanation of how they will be
%used in the agreement. Do not list contractors by name unless the
%cooperator has already bid the work to be done or a sole source
%justification has been provided. A separate budget specifically for the
%sub-contract should be contained in the body of the work plan.
%\textbf{PLEASE NOTE -- Do not list funds used in sub-agreements in
%sections A through D. All funds to be used by cooperators or contractors
%should be outlined in section E. See example below.)}

\begin{description}
	
	\item[Contract with AgResearch New Zealand] Grant funding will be used to support an existing collaboration with Dr. Sean Marshall and Dr. Trevor Jackson who are recognized as global experts on biological control of CRB using OrNV. AgResearch New Zealand will provide molecular diagnostics for genotyping specimens of CRB and samples of OrNV. AgResearch maintains a collection of OrNV isolates in insect cell culture and has the facilities to mass produce virus \textit{in vitro} once we have identified promising candidates for CRB-G biocontrol agents. \textbf{\$35,000}
	
	\item[Cell Phone Contract with Docomo Pacific] We rent a cell phone with unlimited data to be used as a dedicated device for CRB surveys. We commonly use the EpiCollect app for these surveys.  This cell phone also doubles as a safety device for technicians working alone in remote locations. \textbf{\$1,000}
	
	\item[Total for Contracts and Subagreements] \textbf{\$36,000}
	
\end{description}


\subsection{Indirect Cost} 
%(Any resources not discussed in the previous
%categories but identified in the budget should be explained.)

\begin{description}
	
	\item[Administrative fee] (10\% of total grant charged by Research Corporation of the University of Guam)=\textbf{\$20,000}
	
\end{description}

\section{Budget} 

%{[}use whole dollars only\ldots{}no cents. The
%total dollars of the categories below (referred to as budget object
%classes or BOCs by the financial community) is the minimum detail to be
%listed. Some institutions provide detailed financial plans for the
%budget but should be a detailed breakdown of the major categories
%below.{]}


\begin{tabular}{ lr } 
	\hline
	Salaries          &\$86,000\\
	Benefits          &\$20,790\\ 
	Supplies          &\$18,776\\ 
	Travel            &\$18,434\\ 
	Contracts         &\$36,000\\ 
	\hline
	Total Direct Cost &\$180,000\\ 
	Indirect Cost     &\$20,000\\ 
	\hline
	Total Cost        &\$200,000\\ 
	\hline
\end{tabular}

\section{Data} 

%(This section should address what type of data will be
%collected and how will it be maintained. Address timelines for
%collection and recording of data as well as how APHIS will be provided
%access to the data.

All laboratory data is stored in an online laboratory information system (LIMS) which we have custom built for this purpose. Results of bioassays are recorded in technical reports which are also accessible online.

The CRB damage survey component of the project is being developed as an open science project hosted on the Open Science Framework site.

Online access to all project data will be provided to APHIS. In addition links to datasets, technical reports and journal articles will be provided in the semiannual and final reports for this project.

\end{document}
